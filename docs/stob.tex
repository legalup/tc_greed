% A percent sign tells the LaTeX compiler to ignore whatever comes after it.
% So you can use it to make comments in your .tex file.


% We have to tell LaTeX what sort of document we're creating.
% This will be an article, and we're using 11pt font (10 is too small, 12 is too big, imho).
\documentclass[11pt]{amsart}

%\usepackage{natbib}

% This next line forces the compiler to make pdf files. 
\pdfoutput=1

% Next we can include some packages with other features. 
% I have added one that makes your document take up a full page.
% The packages amsmath and amssymb give you a bunch of extra math symbols.
\usepackage{fullpage}
\usepackage{amsmath, amssymb, amsthm}




% The following code defines a ``problem'' environment that I use for homeworks and exams.
% Check out the examples below.
\newtheorem{problem}{Problem}

\newtheorem{theorem}{Theorem}[section]
\newtheorem{corollary}{Corollary}[theorem]
\newtheorem{lemma}[theorem]{Lemma}

\theoremstyle{definition}
\newtheorem{definition}{Definition}[section]


% You can also define your own macros to save on typing time.
% The following macro allows me to type $\ZZ$ instead of 
% $\mathbb{Z}$ in order to produce the symbol for the integers.
\newcommand{\RR}{\mathbb{R}}



%%%%%%%%%%%%%%
%%%%%%%%%%%%%%

% These lines put a heading on the first page. 
\title{Some thoughts on backprop}
\author{Luis Galup}

% If you don't include a date between the braces, it will simply print today's date.
\date{}





%%%%%%%%%%%%%%

% Now we have to tell the compiler that the content of the article is beginning. 
% We also have to tell it to make a title and add our author info.


\begin{document}
\maketitle

In this paper I use einsteinian notation for tensor products.

\begin{definition}[RELU]
\cite{boyd2007tutorial}
RELU $\Re \rightarrow \Re$ is a function where, for $x \in \Re$
\[
	RELU(x) = \max(0,x)
\]
For vectors, we define it coordinatewise.
\end{definition}

We will examine DAG networks. They are defined as follows:

\begin{definition}[DAGN]
Here, $x \in \Re ^ {N}, y \in \Re ^ M$
\begin{align*}
x^{(0)} & = x \\
x^{(1)} & = \phi ^{(1)}(x^{(0)}, w^{(1)}) \\
x^{(2)} & = \phi ^{(2)}(x^{(1)}, x^{(0)}, w^{(2)}) \\
... & = ... \\
x^{(L)} & = y
\end{align*}
where $x$ is input and $y$ is output.
$\phi$ is defined in the following way.
For two vectors of arbitrary length, $x,y$ we define $x:y$ as their concatenation.
For $\phi(x1,x2,...xp,w):\Re^n \rightarrow \Re^m$ denote $X = x1:x2:...:xp$ to be a vector of size n, and denote $w:\Re^n \rightarrow \Re^m$ to be a matrix over the reals. Then
\[
\phi(x1,x2,...xp,w) = RELU(w*X)
\]
\end{definition}
%
%The total input $x_j$ to unit $j$ is a linear function of the outputs, $y_i$, of the units that are connected to $j$ and of the weights $w_{ji}$, on these connections:
%\[
%	x_j = y^iw_{ji}
%\]
%
%Units can be given biases by introducing an extra input to each unit which has the value $1$. The weight on this extra input is called the bias and is equivalent to a threshold of the opposite sign. It can be treated just like other weights.
%
%A unit has a real valued out, $y_j$, which is a nonlinear function of its total input.
%\[
%	y_j = RELU(x_j)
%\]

We define the residual error, E, as
\[
	E_{j,c} = y_{j,c} - d_{j,c}
\]
where $c$ is an index over input-output pairs, and j is an index over output units, and y is the actual stated of the an output unit and d is its desired state. 
The total error (loss) $L$ is defined as the L2 norm of the residual,
\[
	L = \| E \|_2
\]

It is at this point that backpropagation is used to define to update the weights to minimize this. 

Taking derivatives is a good idea when the derivatives are well defined and far from being singular. However, this is difficult to control and is subject to the torments of local minima.


We introduce a new backpropagation technique which does not use derivatives, but instead uses quadratic programming. 

% this is taken from boyd, vanderberghe, page 152
The general quadratic program (QP) can be stated as 
\begin{definition}
\begin{align*}
\min_x q(x)  & =  \frac{1}{2} x^T P x + c^Tx + r\\
\hbox{subject to } Gx & < h \\
\hbox{and } Ax & = b
\end{align*}
\end{definition}

%
%\begin{theorem}[Pythagorean theorem]
%\label{pythagorean}
%This is a theorema about right triangles and can be summarised in the next 
%equation 
%\[ x^2 + y^2 = z^2 \]
%\end{theorem}
%
%And a consequence of theorem \ref{pythagorean} is the statement in the next 
%corollary.
%
%\begin{corollary}
%There's no right rectangle whose sides measure 3cm, 4cm, and 6cm.
%\end{corollary}
%
%You can reference theorems such as \ref{pythagorean} when a label is assigned.
%
%\begin{lemma}
%Given two line segments whose lengths are $a$ and $b$ respectively there is a 
%real number $r$ such that $b=ra$.
%\end{lemma}
%%


% To tell the compiler when to stop working, you have to include the following line at the end.
\bibliographystyle{plain}
\bibliography{refs}
\end{document}
